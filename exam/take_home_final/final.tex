\documentclass{report}

\usepackage{fullpage}
\usepackage[skip=4pt]{caption} % ``skip'' sets the spacing between the figure and the caption.
\usepackage{tikz}
\usepackage{pgfplots}   % Needed for plotting
\pgfplotsset{compat=newest}
\usepgfplotslibrary{fillbetween}  % Allow for highlighting under a curve
\usepackage{amsmath}    % Allows for piecewise functions using the ``cases'' construct
\newcommand{\abs}[1]{\lvert#1\rvert}%
\usepackage{mathtools} % for "\DeclarePairedDelimiter" macro
\DeclarePairedDelimiter{\floor}{\lfloor}{\rfloor}
\DeclarePairedDelimiter{\ceil}{\lceil}{\rceil}
%\usepackage{mathrsfs}   % Use the ``\mathscr'' command in an equation.
\usepackage{siunitx}    % Allows for ``S'' alignment in table to align by decimal point
\usepackage{enumitem}   % Reference enumerated item lists.

\usepackage[obeyspaces,spaces]{url} % Used for typesetting with the ``path'' command
\usepackage[hidelinks]{hyperref}   % Make the cross references clickable hyperlinks
\usepackage[bottom]{footmisc} % Prevents the table going below the footnote
\usepackage{nccmath}    % Needed in the workaround for the ``aligncustom'' environment
\usepackage{amssymb}    % Used for black QED symbol
\usepackage{bm}    % Allows for bolding math symbols.
\usepackage{tabto}     % Allows to tab to certain point on a line
\usepackage{float}
\usepackage{multirow}

\newcommand{\hangindentdistance}{1cm}
\newcommand{\defaultleftmargin}{0.25in}
\newcommand{\questionleftmargin}{-.5in}
%\setlength{\parindent}{0pt}
\setlength{\parskip}{1em}
\setlength{\oddsidemargin}{\defaultleftmargin}
\usepackage{scrextend}
\newcommand{\eref}[1] {(\ref{#1})}


% Set up page formatting
\usepackage{todonotes}
\usepackage{fancyhdr} % Used for every page footer and title.
\pagestyle{fancy}
\fancyhf{} % Clears both the header and footer
\renewcommand{\headrulewidth}{0pt} % Eliminates line at the top of the page.
\fancyfoot[LO]{CMPS218 \textendash\ Take Home Final} % Left
\fancyfoot[CO]{\thepage} % Center
\fancyfoot[RO]{Zayd Hammoudeh} %Right

% Change interline spacing.
\renewcommand{\baselinestretch}{1.1}
\newenvironment{aligncustom}
{ \csname align*\endcsname % Need to do this instead of \begin{align*} because of LaTeX bug.
    \centering
}
{
  \csname endalign*\endcsname
}
%--------------------------------------------------


\title{\textbf{CMPS218 \textendash\ Take Home Final Exam}}
\author{Zayd Hammoudeh}

%---------------------------------------------------%
% Define the Environments for the Problem Inclusion %
%---------------------------------------------------%
\newcounter{subProbCount}       % Initialize the subproblem counter
\newcounter{problemCount}
\setcounter{problemCount}{0} % Reset the subproblem counter
\newenvironment{problemshell}{
  \begin{addmargin}[\questionleftmargin]{0em}
    \par%
    \medskip
    \leftskip=0pt\rightskip=0pt%
    \setlength{\parindent}{0pt}
    \bfseries
}
{
    \par\medskip
  \end{addmargin}
}
\newenvironment{problem}[2]
{%
  \begin{problemshell}
    \noindent \textit{Chapter~\##1, Exercise~\##2} \\
}
{
  \end{problemshell}
  \setcounter{subProbCount}{1} % Reset the subproblem counter
}
\newenvironment{subproblem}
{%
  \begin{problemshell}
    \setlength{\leftskip}{\hangindentdistance}
    % Print the subproblem count and offset to the left
    \hspace{-\hangindentdistance}(\alph{subProbCount}) \tabto{0pt}
}
{
    \stepcounter{subProbCount} % Increment the subproblem counter
  \end{problemshell}
}


\begin{document}
%  \maketitle
%
  \noindent
  \textbf{Name}: Zayd Hammoudeh

  \noindent
  \textbf{Assignment}: CMPS218 Take Home Final Exam

  \noindent
  \textbf{Other Student Discussions}: Below is a summary of my discussions with other students for this exam.
  \vspace{-1.5em}
  \begin{itemize}
    \setlength\itemsep{0em}
    \item \textit{Problem~\#1}: This problem was discussed with Konstantinos Zampetakis, Will Bolden, and Noujan Pashanasangi.

    \item \textit{Problem~\#5}: The Seven Scientists -- This problem originally appeared in the homework assignment.  At the time, I discussed it with William Bolden, Konstantinos Zampetakis, and Keller Jordan.  My answer is identical to the homework other than perhaps a couple of grammar corrections.
  \end{itemize}

  \noindent
  \textbf{Problem Assessments}: All problems are complete to the best of my understanding.  I am not fully satisfied with my answer to the first problem on the twelve~balls.  It was quite challenging, and I am not sure how well my estimate aligns with the true number of weights required.

  \begin{problem}{4}{14}
  You are given 12 balls and the three-outcome balance of exercise 4.1; this time, two of the balls are odd; each odd ball may be heavy or light, and we don't know which.  We want to identify the odd balls and in which direction they are odd.
\end{problem}

\begin{subproblem}
  \textit{Estimate} how many weights are required by the optimal strategy.  And what if there are three odd balls.
\end{subproblem}

\begin{subproblem}
  How do your answers change if it is known that all the regular balls weigh 100g, that light balls weight 99g, and heavy ones weigh 110g.
\end{subproblem}




  \newpage
\begin{problem}{5}{8}{30}
  \textit{Scientific American} carried the following puzzle in 1975.
  \problemspace
  \textbf{The poisoned glass:} \textnormal{\textit{'Mathematicians are curious birds,' the police commissioner said to his wife. 'You see, we had all those partly filled glasses lined up in rows on a table in the hotel kitchen. Only one contained poison, and we wanted to know which one before searching the glass for fingerprints.  Our lab could test the liquid in each glass, but the tests take time and money, so we wanted to make as few of them as possible by simultaneously testing mixtures of small samples from groups of glasses.  The university sent over a mathematics professor to help us.  He counted the glasses, smiled and said: \\ ``Pick any glass you want, Commissioner. We'll test it first.'' \\ ``But won't that waste a test?'' I asked. \\ ``No,'' he said. ``it's part of the best procedure.  We can test one glass first.  It doesn't matter which one.'' \\ `How many glasses were there to start with?' the commissioner's wife asked. \\ 'I don't remember. Somewhere between 100 and 200.'} \\What was the exact number of glasses?}
  \problemspace
  Solve this puzzle and then explain why the professor is in fact wrong and the commissioner was right.  What is in fact the optimal procedure for identifying the one poisoned glass?  What is the expected waste relative to this optimum if one followed the professor's strategy?  Explain the relationship to symbol coding.
\end{problem}

The test for poison has a binary outcome, i.e.,~the sample either has poison or not.  Therefore, assuming each cup has poison with equal probability, the size of the remaining set of glasses is, on average, cut in half with each test.

If the number of glasses,~$n$, is a power of~$2$, then the number of tests required is $\lg n$, where $\lg$ is the base-$2$ logarithm.  Note that the only power of~2 between 100 and 200 is 128.  There was one extra glass that the professor tested separately.  Therefore, there was \boxed{129~\text{glasses}}.

\begin{table}[h]
  \centering
  \begin{tabular}{c|c|c}
    \hline
    Glass ID & Probability of Poison & \# Tests  \\\hline
    1        & 1/129                 & 1         \\\hline
    2-129    & 128/129               & 1 + 7 = 8 \\\hline
  \end{tabular}
  \caption{Number of tests required using the professor's strategy}\label{tab:problem5.9.20-Prof}
\end{table}

Table~\ref{tab:problem5.9.20-Prof} shows the number of tests required when using the professor's strategy.  Glass~\#1 represents the first glass tested, i.e.,~the one selected at random.  In the unlikely event that glass has the poison, no additional testing is required. In contrast, if the poison is in one of the other 128~glasses, seven tests (plus the additional one for the first glass) are required.  Using this strategy, the expected number of tests is:

\begin{aligncustom}
  \mathbb{E}(\text{Professor's Strategy}) &= \frac{1}{129} \cdot 1 + \frac{128}{129} \cdot 8 \\
                                          &\approx \boxed{7.946}\text{.}
\end{aligncustom}


In contrast, the optimal strategy is:
\vspace{-1.25em}
\begin{enumerate}[leftmargin=1.35cm]
  \setlength\itemsep{0em}
  \item Select one glass at random and leave it off to the side.
  \item\label{itm:test} Test a sample that combines wine from half of the remaining glasses (excluding the one off to the side).
  \item\label{itm:discard} If poison is observed in this tested sample, discard the untested glasses.  Otherwise, discard the tested glasses.
  \item Repeat steps~\#\ref{itm:test} and~\#\ref{itm:discard} until only a single glass remains (excluding the one off to the side).
  \item If poison was ever observed in any of the previous tests, then the remaining glass not off to the side has the poison, and no additional testing is required.
  \item\label{itm:testRemaining} If poison was never observed in any test, then test just the remaining glass not off to the side. If poison is detected, then the answer is clear, and the tested glass has poison; otherwise, the glass off to the side has the poison.
\end{enumerate}

Table~\ref{tab:problem5.9.20-Opt} shows the number of tests required using this optimum strategy.    Note that Glass~\#129 entails the remaining glass in step~\#\ref{itm:testRemaining} where no sample tests positive for poison up to the last remaining glass not off to the side.

\begin{table}[h]
  \centering
  \begin{tabular}{c|c|c}
    \hline
    Glass ID & Probability of Poison & \# Tests  \\\hline
    1        & 1/129                 & 7 + 1     \\\hline
    2-128    & 127/129               & 7         \\\hline
    129      & 1/129                 & 7 + 1     \\\hline
  \end{tabular}
  \caption{Number of tests required using the optimum strategy}\label{tab:problem5.9.20-Opt}
\end{table}

\noindent
Using this optimum strategy, the expected number of tests is:

\begin{aligncustom}
  \mathbb{E}(\text{Optimum Strategy}) &= \frac{2}{129} \cdot 8 + \frac{127}{129} \cdot 7 \\
  &\approx \boxed{7.016}\text{.}
\end{aligncustom}

It is clear then that the expected waste of the professor's strategy is \boxed{0.93} tests.

Maximum compression of a symbol code is achieved by assigning shorter codes (i.e.,~with less bits) to outcomes with higher probability.  In contrast, the professor prioritized the least likely outcome by testing the randomly selected glass first.  The optimal strategy described above always tests the most likely outcome (i.e.,~more glasses at once) similar to how symbol codes are encoded.

  \newpage
\begin{problem}{20}{2}{2}
  Show that as the stiffness~$\beta$ goes to~$\infty$, the soft K-means algorithm becomes identical to the original hard K-means algorithm except for the way in which means assigned no points behave.  Describe what those means do instead of sitting still.
\end{problem}

In the standard or ``hard'' K-means algorithm, each point is assigned to exactly one cluster.  As such, each of a cluster's points have equal membership.

Soft K-means reduces the rigidity of the standard K-means by introducing a new ``stiffness'' hyperparameter,~$\beta$.  Rather than each point being a member of exclusively one cluster, the \textit{responsibility} for each point is shared (generally unevenly) among all $K$~clusters.  For cluster~$k\in\{1,\ldots,K\}$ and point~$\textbf{x}^{(n)}$, the responsibility,~$r_{k}^{(n)}$, is:

\[ r_{k}^{(n)} = \frac{\exp(-\beta d(\textbf{m}^{k}, \textbf{x}^{(n)}))}{\sum_{k'} \exp(-\beta d(\textbf{m}^{k'}, \textbf{x}^{(n)}))} \]

\noindent
where $d$ is the distance metric, and $\textbf{m}^{k}$ is the center of cluster~$k$.

As $\beta$ increases, then even small differences in $d$ can cause massive changes in responsibility.  As ${\beta \rightarrow \infty}$, all responsibility for a point will be assigned to its nearest cluster.  This behavior is exactly the same as standard, ``hard'' K-means where points belong to only the cluster's whose centroid is closest.

Considering the second part of the question, the centroid update rule for cluster,~$k$, is:

\[ \mathbf{m}^{(k)} = \frac{\sum_{n}r^{(n)}_k \mathbf{x}^{(n)}}{R^{(k)}} \]

\noindent
where the total responsibility, $R^{(k)}$, for cluster $k$ is:

\[ R^{(k)}=\sum_{n} r_{k}^{(n)} \text{.} \]

As mentioned previously, small differences in the distance,~$d$, cause huge differences in the responsibility,~$r$, when $\beta$ approaches infinity.  In hard K-means, it was possible for a cluster to be assigned no points.  However, in soft K-Means, such clusters still have \textit{some} responsibility for every point, albeit infinitesimally small when $\beta$ approaches infinity.  In that case, those means will move to the \textit{location of the point with the highest responsibility, which may not necessarily be the closest to that mean}.


  \newpage
\begin{problem}{15}{6}
  How does \textit{this} trick work?
  
  \textnormal{\textit{`Here's an ordinary pack of cards shuffled into random order.  Please choose five cards from the pack, any you wish. Don't let me see their faces. No, don't give them to me: pass them to my assistant Esmerelda.  She can look at them.}}
  
  \textnormal{\textit{`Now, Esmerelda, show me four of the cards.  Hmm...nine of spades, six of clubs, four of hearts, ten of diamonds. The hidden card must be the queen of spades.}}
  
  This trick can be performed as described above for a pack of 52 cards.  Use information theory to give an upper bound on the number of cards for which the trick can be performed.
\end{problem}

A set of unique objects (e.g.,~cards in a deck) can be bijectively mapped to the set of integers $\{1,\ldots,n\}$ where $n$ is the cardinality of the set. If we try to use the same trick as in the previous problem, then we will run into an issue since there would only be four letters (one for each of the revealed cards) and ${}_{4}P_{4} = 4! = 24$ permutations.  If $n=52$, this is not enough permutations to cover the whole deck so we need to get more creative....




  \include{p5}

\end{document}

