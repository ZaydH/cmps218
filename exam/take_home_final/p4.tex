\begin{problem}{15}{6}
  How does \textit{this} trick work?

  \begin{addmargin}[0.1in]{0.1in}
    \textnormal{\textit{`Here's an ordinary pack of cards shuffled into random order.  Please choose five cards from the pack, any you wish. Don't let me see their faces. No, don't give them to me: pass them to my assistant Esmerelda.  She can look at them.}}

    \textnormal{\textit{`Now, Esmerelda, show me four of the cards.  Hmm...nine of spades, six of clubs, four of hearts, ten of diamonds. The hidden card must be the queen of spades.}}
  \end{addmargin}

  This trick can be performed as described above for a pack of 52 cards.  Use information theory to give an upper bound on the number of cards for which the trick can be performed.
\end{problem}

\noindent
\textit{Assumption}:

To make this problem more meaningful, I am going to assume that when Esmerelda ``shows'' the four cards to the magician, she is not using any sleight of hand trickery to relay additional information to the magician (e.g., orient the cards in her hand in a certain way, show the face or backside first, etc.).  Essentially, I consider that the magician is only provided numbers from a black box.

\noindent
\textit{Answer}:

Consider a collection of $n$~unique objects.  It is trivial to bijectively map these objects to the integers $\{1,\ldots,n\}$. If five integers from the set are selected of which four are revealed, then the previous problem's trick allows us to communicate the fifth object as long as $n$~is less than or equal to ${4! + 4 = 28}$.  A standard deck contains ${n=52}$ cards so this is clearly insufficient to uniquely encode the hidden, fifth card.  Therefore, we need a more creative encoding....

The key to this problem is that Esmerelda \textit{chose} which of the five cards remained hidden.  If she chose that card randomly, we are back to 24~permutations.  However, Esmerelda was more cunning and chose a \textit{specific} card to remain hidden.  A scheme could be developed that uses this selection process to increase the upper bound on the number of encodable objects fivefold i.e.,~once for each card she could have chosen.  When we combine that knowledge with the technique from the last question (exercise~15.5), we see that the maximum number of objects that can be uniquely encoded is~${5\cdot4!=120}$.  Therefore, combining the maximum size of the encodable objects pool with the four cards Esmerelda revealed, we see the upper bound for the maximum deck size is~${4 + 120 = \boxed{124}}$.

The problem does not require that we prove the tightness of this upper bound.  However, after completing the problem, I did research online and found that this is in fact the true bound.  What is more, the technique described above is not how this trick (known as the ``Fitch Cheney Card Trick'') is traditionally performed; that approach to the trick has a maximum deck size of~52.
