\begin{problem}{15}{6}
  How does \textit{this} trick work?

  \textnormal{\textit{`Here's an ordinary pack of cards shuffled into random order.  Please choose five cards from the pack, any you wish. Don't let me see their faces. No, don't give them to me: pass them to my assistant Esmerelda.  She can look at them.}}

  \textnormal{\textit{`Now, Esmerelda, show me four of the cards.  Hmm...nine of spades, six of clubs, four of hearts, ten of diamonds. The hidden card must be the queen of spades.}}

  This trick can be performed as described above for a pack of 52 cards.  Use information theory to give an upper bound on the number of cards for which the trick can be performed.
\end{problem}

Consider a collection of $n$~unique objects.  It is trivial to bijectively map these objects to the integers $\{1,\ldots,n\}$. If five integers from the set are selected of which four are revealed, then the previous problem's trick would allow us to communicate the fifth object as long as $n$~is less than or equal to ${4! + 4 = 28}$.  Given a standard deck contains $n=52$ cards, the number of permutations is insufficient to uniquely encode the hidden card.  Therefore, we need a more creative encoding....

The key to this problem is that Esmerelda \textit{chose} which of the five cards will remain hidden.  If she chose that card randomly, we are back to 24~permutations.  However, Esmerelda was more cunning and chose a \textit{specific} card to remain hidden.  A scheme could be developed that uses this selection process to increase the upper bound on the number of encodable objects fivefold i.e.,~once for each card the volunteer chose.  When we combine that with the trick from the previous problem, we see that the maximum number of objects that can be uniquely encoded is now~${5\cdot4!=120}$.  Therefore, combining the maximum size of the encoded objects pool with the four cards revealed, we see the upper bound for the maximum deck size is~${4 + 120 = \boxed{124}}$.

The problem does not require that we prove the upper bound is the true bound.  However, after completing the problem, I did research online and found that this is in fact the true bound.
