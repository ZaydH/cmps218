\newpage
\begin{problem}{15}{6}
  How does \textit{this} trick work?
  
  \textnormal{\textit{'Here's an ordinary pack of cards shuffled into random order.  Please choose five cards from the pack, any you wish. Don't let me see their faces. No, don't give them to me: pass them to my assistant Esmerelda.  She can look at them.}}
  
  \textnormal{\textit{'Now, Esmerelda, show me four of the cards.  Hmm...nine of spades, six of clubs, four of hearts, ten of diamonds. The hidden card must be the queen of spades.}}
  
  This trick can be performed as described above for a pack of 52 cards.  Use information theory to give an upper bound on the number of cards for which the trick can be performed.
\end{problem}

A deck of cards is simply a set of unique items upon which a total ordering can be applied.  

\textit{Example Total Ordering:} All hearts are less than all diamonds which are less than all spades which are less than all clubs.  Cards of the \textit{same suit} are ordered according to poker rules, e.g.,

\[2 < 3 < \ldots < 10 < \text{Jack} < \text{Queen} < \text{King} < \text{Ace} \text{.} \]

Given a total ordering, the cards can be bijectively mapped to the set of integers $\{1,\ldots,n\}$ where $n$ is the number of cards in the deck. However, simply using the trick from the previous problem will not work since there would only be four letters (one for each of the revealed cards) and ${}_{4}P_{4} = 4! = 24$ permutations.  If $n=52$, this is not enough permutations to cover the whole deck so we need to get more creative....


