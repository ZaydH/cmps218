\begin{problem}{4}{14}
  You are given 12 balls and the three-outcome balance of exercise 4.1; this time, two of the balls are odd; each odd ball may be heavy or light, and we don't know which.  We want to identify the odd balls and in which direction they are odd.
\end{problem}



\begin{subproblem}
  \textit{Estimate} how many weights are required by the optimal strategy.  And what if there are three odd balls?
\end{subproblem}

  When there was a single odd ball out of twelve, the size of the hypothesis class was ${2 * \binom{12}{1} = 24}$.  To determine the number of weights required to determine the actual solution, we need to consider how each hypothesis would be represented on the balance.  The six possibilities are shown in Table~\ref{tab:p1OneBallTable}.  As enumerated in Figure~4.2 of Mackay's text, we see that balance output has \textit{roughly} equivalent probability making the number of required weights ${\ceil{\log_{3} 24}=3}$.
  
  \begin{table}[H]
    \centering
    \caption{Three-outcome balance position for a \\single odd ball that is light or heavy ball}\label{tab:p1OneBallTable}
    \begin{tabular}{|c||c|c||c|}
      \hline
      \multirow{2}{*}{Case~\#} &  \multicolumn{2}{c||}{Odd Ball} & Balance\\\cline{2-3}
          &  Weight  &  Location &  Position  \\ \hline\hline
      1   &  Light   &  Left     &  $\bar{~}\bar{~}|\underline{~~}$ \\ \hline
      2   &  Heavy   &  Right    &  $\bar{~}\bar{~}|\underline{~~}$ \\ \hline\hline
      3   &  Light   &  Right    &  $\underline{~~}|\bar{~}\bar{~}$ \\ \hline
      4   &  Heavy   &  Left     &  $\underline{~~}|\bar{~}\bar{~}$ \\ \hline\hline
      5   &  Light   &  Neither  &  $-|-$ \\ \hline
      6   &  Heavy   &  Neither  &  $-|-$ \\ \hline
    \end{tabular}
  \end{table}

  \noindent
  
  The situation grows significantly more complex when there are three balls.  For example, the size of the hypothesis class grows by more than an order of magnitude from~$24$ to ${2\cdot2\cdot\binom{12}{2} = 264}.$ In addition, interpreting the results of the balance also becomes more complex as shown in Table~\ref{tab:p1TwoBallsUnknownWeights}.  Note first that the possible cases has increased six-fold from~6 to~36.  Furthermore, as part of the problem, we are not provided any information about the relative or absolute weights of the balls.  Therefore, we do not know if for example a heavy ball cancels out a light ball.  Such additional uncertainty necessitates that we collect more require more weights to achieve certainty.
  
  \begin{table}[H]
    \centering
    \caption{Three-outcome balance position for two\\odd balls that are light and/or heavy Ball}\label{tab:p1TwoBallsUnknownWeights}
    \begin{tabular}{|c||c|c||c|c||c|}
      \hline
      \multirow{2}{*}{Case~\#} & \multicolumn{2}{c||}{Odd Ball~\#1} &  \multicolumn{2}{c||}{Odd Ball~\#2} & Balance\\\cline{2-5}
          &  Weight  &  Location &  Weight  &  Location &  Position  \\ \hline\hline
      1   &  Light   &  Left     &  Light   &  Left     &  $\bar{~}\bar{~}|\underline{~~}$ \\ \hline
      2   &  Light   &  Left     &  Light   &  Neither  &  $\bar{~}\bar{~}|\underline{~~}$ \\ \hline
      3   &  Light   &  Left     &  Heavy   &  Neither  &  $\bar{~}\bar{~}|\underline{~~}$ \\ \hline
      4   &  Light   &  Neither  &  Light   &  Left     &  $\bar{~}\bar{~}|\underline{~~}$ \\ \hline
      5   &  Heavy   &  Neither  &  Light   &  Left     &  $\bar{~}\bar{~}|\underline{~~}$ \\ \hline
      6   &  Heavy   &  Right    &  Heavy   &  Right    &  $\bar{~}\bar{~}|\underline{~~}$ \\ \hline
      7   &  Heavy   &  Right    &  Light   &  Neither  &  $\bar{~}\bar{~}|\underline{~~}$ \\ \hline
      8   &  Heavy   &  Right    &  Heavy   &  Neither  &  $\bar{~}\bar{~}|\underline{~~}$ \\ \hline
      9   &  Light   &  Neither  &  Heavy   &  Right    &  $\bar{~}\bar{~}|\underline{~~}$ \\ \hline
      10  &  Heavy   &  Neither  &  Heavy   &  Right    &  $\bar{~}\bar{~}|\underline{~~}$ \\ \hline
      11  &  Light   &  Right    &  Heavy   &  Left     &  $\bar{~}\bar{~}|\underline{~~}$ \\ \hline
      12  &  Heavy   &  Left     &  Light   &  Left     &  $\bar{~}\bar{~}|\underline{~~}$ \\ \hline\hline
         
      13  &  Light   &  Right    &  Light   &  Right    &  $\underline{~~}|\bar{~}\bar{~}$ \\ \hline
      14  &  Light   &  Right    &  Light   &  Neither  &  $\underline{~~}|\bar{~}\bar{~}$ \\ \hline
      15  &  Light   &  Right    &  Heavy   &  Neither  &  $\underline{~~}|\bar{~}\bar{~}$ \\ \hline
      16  &  Light   &  Neither  &  Light   &  Right    &  $\underline{~~}|\bar{~}\bar{~}$ \\ \hline
      17  &  Heavy   &  Neither  &  Light   &  Right    &  $\underline{~~}|\bar{~}\bar{~}$ \\ \hline
      18  &  Heavy   &  Left     &  Heavy   &  Left     &  $\underline{~~}|\bar{~}\bar{~}$ \\ \hline
      19  &  Heavy   &  Left     &  Light   &  Neither  &  $\underline{~~}|\bar{~}\bar{~}$ \\ \hline
      20  &  Heavy   &  Left     &  Heavy   &  Neither  &  $\underline{~~}|\bar{~}\bar{~}$ \\ \hline
      21  &  Light   &  Neither  &  Heavy   &  Left     &  $\underline{~~}|\bar{~}\bar{~}$ \\ \hline
      22  &  Heavy   &  Neither  &  Heavy   &  Left     &  $\underline{~~}|\bar{~}\bar{~}$ \\ \hline
      23  &  Light   &  Left     &  Heavy   &  Right    &  $\underline{~~}|\bar{~}\bar{~}$ \\ \hline
      34  &  Heavy   &  Right    &  Light   &  Right    &  $\underline{~~}|\bar{~}\bar{~}$ \\ \hline\hline

      25  &  Light   &  Right    &  Light   &  Left     &  $-|-$ \\ \hline
      26  &  Light   &  Left     &  Light   &  Right    &  $-|-$ \\ \hline
      27  &  Heavy   &  Right    &  Heavy   &  Left     &  $-|-$ \\ \hline
      28  &  Heavy   &  Left     &  Heavy   &  Right    &  $-|-$ \\ \hline
      29  &  Light   &  Neither  &  Light   &  Neither  &  $-|-$ \\ \hline
      30  &  Light   &  Neither  &  Heavy   &  Neither  &  $-|-$ \\ \hline
      31  &  Heavy   &  Neither  &  Light   &  Neither  &  $-|-$ \\ \hline
      32  &  Heavy   &  Neither  &  Heavy   &  Neither  &  $-|-$ \\ \hline\hline

      33  &  Light   &  Left     &  Heavy   &  Left    &  \textbf{Unknown} \\ \hline
      34  &  Heavy   &  Left     &  Right   &  Left    &  \textbf{Unknown} \\ \hline
      35  &  Light   &  Right    &  Heavy   &  Right   &  \textbf{Unknown} \\ \hline
      36  &  Heavy   &  Right    &  Light   &  Right   &  \textbf{Unknown} \\ \hline
    \end{tabular}
  \end{table}
  
  When there are three odd balls, the uncertainty increases further.  First, the hypothesis class grows from~$264$ to~${2\cdot2\cdot2\cdot\binom{12}{3} = 1,760}$.  Next, the table of cases (not included) grows from~$36$ to~$216$.  Furthermore, the cases where the balance output would correspond to ``Unknown'' grows from~$4$ to~$24$ as shown in Table~\ref{tab:p1ThreeBallsUnknownWeights}.
  
  \begin{table}[H]
    \centering
    \caption{Three odd ball cases where the balance position is unknown if ball weights are not provided where an ``X'' corresponds to ``do not care''}\label{tab:p1ThreeBallsUnknownWeights}
    \begin{tabular}{|c||c|c||c|c||c|c||}
      \hline
      \multirow{2}{*}{Case~\#} & \multicolumn{2}{c||}{Odd Ball~\#A} & \multicolumn{2}{c||}{Odd Ball~\#B} & \multicolumn{2}{c||}{Odd Ball~\#C}\\\cline{2-7}
          &  Weight  &  Location &  Weight  &  Location &  Weight  &  Location  \\ \hline\hline
      
      1--6   &  Light &  Left  &  Heavy &  Left  &  X &  X   \\ \hline
      7--12  &  Heavy &  Left  &  Light &  Left  &  X &  X   \\ \hline
      13--18 &  Light &  Right &  Heavy &  Right &  X &  X   \\ \hline
      19--24 &  Heavy &  Right &  Light &  Right &  X &  X   \\ \hline
    \end{tabular}
  \end{table}
  
  Now consider when there are two odd balls.  There are substantially more possible outcomes.  Likewise, since the relative weights of the light and heavy balls is unknown, there is also some ambiguity con

  From a set of 12 balls, there are $\binom{12}{2}$ combinations of two ``odd'' balls.  Similarly, each ``odd'' ball can be either light or heavy.  Therefore, the number of unique solutions is:

  \[ 2\cdot 2 \cdot \binom{12}{2} = 4 \cdot 66 = 264 \text{,} \]

  \noindent
  all of which are equally likely.  Ideally, the three-outcome balance roughly reduces the set of possible solutions by two-thirds with each measurement.  A best case expected number of weights for two odd balls is:

  \[ \log_{3} 264\approx 5.075 {.} \]

  Since 5.075 is much closer to~5 than~6, there is still slack for imbalance in the probabilities of the respective outcomes of a ball weighing.  Therefore, I estimate \boxed{6}~weights are required for two odd balls.

  When there are three odd balls, then there are $\binom{12}{3}$ combinations of ``odd'' balls.  Again, since each odd ball is either light or heavy, the number of unique (equally likely) solutions is:

  \[ 2^3 \cdot \binom{12}{3} = 8 \cdot 220 = 1760 \text{.} \]

  \noindent
  As such, the minimum expected number of weightings for three odd balls is:

  \[ \log_{3} 1760 \approx 6.802 \text{.} \]

  \noindent
  Unfortunately, $6.802$ is close to $7$ leaving less slack for any imbalance in the outcome probabilities.  For that reason, I estimate the expected number of weights for three balls is \boxed{7 \text{ or } 8}.

\begin{subproblem}
  How do your answers change if it is known that all the regular balls weigh 100g, that light balls weight 99g, and heavy ones weigh 110g?
\end{subproblem}

  As mentioned previously, in the multi-oddball case, uncertainty about the relative weights of the balls necessitates that additional measurements be made to clarify that uncertainty.  Now that the relative ball weights are known, the Cases~\#33-34 and Cases~\#35-36 in Table~\ref{tab:p1TwoBallsUnknownWeights} are assigned to ``${\underline{~~}|\bar{~}\bar{~}}$'' and ``${\bar{~}\bar{~}|\underline{~~}}$'' respectively.