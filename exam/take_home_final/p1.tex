\begin{problem}{4}{14}
  You are given 12 balls and the three-outcome balance of exercise 4.1; this time, two of the balls are odd; each odd ball may be heavy or light, and we don't know which.  We want to identify the odd balls and in which direction they are odd.
\end{problem}

  \noindent
  \textit{Assumptions}:

  \begin{itemize}
    \item The problem states, ``each odd ball may be heavy or light.'' This language implies that all heavy odd balls have equivalent weight.  Similarly, the weights of all light odd balls are equivalent.
  \end{itemize}

  \noindent
  \textit{Answer}:

  The hypothesis class,~$\mathcal{H}$, is the set of all valid allocations of odd ball(s) in the problem. Given a three-outcome balance, the \textit{lower bound} for the required number of weights is:

  \begin{equation}
    \Omega\left[\#\text{Weights}\right] = \ceil*{\log_{3} \abs{\mathcal{H}}} \text{.}
  \end{equation}

  \noindent
  The \textit{expected} number of weights can be estimated via:

  \begin{equation}\label{eq:p1NumbWeightsRequired}
    \mathbb{E}\left[\#\text{Weights}\right] \approx \ceil*{\frac{\log_{2} \abs{\mathcal{H}}}{H}} \text{,}
  \end{equation}

  \noindent
  where $H$ is the binary entropy of the balance outcomes.  Eq.~\eref{eq:p1NumbWeightsRequired}'s correctness is quite intuitive. Representing each member of~$\mathcal{H}$ requires ${\log_{2}\abs{\mathcal{H}}}$~bits.  Binary entropy is the \textit{expected} information gain per weighing (in bits).  The ratio of the two in turn yields the \textit{expected number of weights}.

  The quality of the estimate Eq.~\eref{eq:p1NumbWeightsRequired} provides is built on two assumptions.  First, since entropy~$H$ is used as a normalizer, the equation assumes that~$H$ remains constant in each weighing.  Next, if the probability distribution associated with the balance outcomes is far from uniform, then there will be a large divergence between this expected quantity and the upper bound (which is more relevant for Mackay's question).

  To verify the Eq.~\eref{eq:p1NumbWeightsRequired}'s applicability, consider the case of a single odd ball.  The cardinality of~$\mathcal{H}$ is ${2 * \binom{12}{1} = 24}$. Each outcome of the balance partitions~$\mathcal{H}$ into subsets of roughly equal probability. This is illustrated in Table~\ref{tab:p1OneBallTable} which shows the correspondence between the odd ball's placement and the balance outcome. Using Eq.~\eref{eq:p1NumbWeightsRequired}, we estimate the expected number of weights required is approximately

  \[ \ceil*{\frac{\log_{2}24}{1.585}} \approx \ceil*{\frac{4.584}{1.585}} \approx \ceil{2.893} = 3 \text{.} \]

  \noindent
  This estimate equals the actual number of weights as shown in Figure~4.2 of Mackay's text.

  \begin{table}[H]
    \centering
    \caption{Three-outcome balance position for a \\single odd ball that is light or heavy}\label{tab:p1OneBallTable}
    \begin{tabular}{|c||c|c||c|}
      \hline
      \multirow{2}{*}{Case~\#} &  \multicolumn{2}{c||}{Odd Ball} & Balance\\\cline{2-3}
      &  Weight  &  Location &  Position  \\ \hline\hline
      1   &  Light   &  Left     &  $\bar{~}\bar{~}|\underline{~~}$ \\ \hline
      2   &  Heavy   &  Right    &  $\bar{~}\bar{~}|\underline{~~}$ \\ \hline\hline
      3   &  Light   &  Right    &  $\underline{~~}|\bar{~}\bar{~}$ \\ \hline
      4   &  Heavy   &  Left     &  $\underline{~~}|\bar{~}\bar{~}$ \\ \hline\hline
      5   &  Light   &  Neither  &  $-|-$ \\ \hline
      6   &  Heavy   &  Neither  &  $-|-$ \\ \hline
    \end{tabular}
  \end{table}


\begin{subproblem}
  \textit{Estimate} how many weights are required by the optimal strategy.  And what if there are three odd balls?
\end{subproblem}

  When there are two odd balls, $\abs{\mathcal{H}}$ is ${2\cdot2\cdot\binom{12}{2} = 264}$. Table~\ref{tab:p1TwoBallsUnknownWeights} shows the possible balance locations for two balls.  Since this part of the problem provides no information about the relative ball weights, it is not possible to know \textit{a priori} the balance's behavior when there are mismatched balls on the same side of the balance.

  \begin{table}
  \centering
  \caption{Three-outcome balance position for two\\odd balls that are light and/or heavy Ball}\label{tab:p1TwoBallsUnknownWeights}
  \begin{tabular}{|c||c|c||c|c||c||c|}
    \hline
    \multirow{2}{*}{Case~\#} & \multicolumn{2}{c||}{Odd Ball~\#1} &  \multicolumn{2}{c||}{Odd Ball~\#2} & Balance & \multirow{2}{*}{$\Pr$}\\\cline{2-5}
    &  Weight  &  Location &  Weight  &  Location &  Position                        &         \\ \hline\hline
    1   &  Light   &  Left     &  Light   &  Left     &  $\bar{~}\bar{~}|\underline{~~}$ & $P_{1}$ \\ \hline
    2   &  Light   &  Left     &  Light   &  Neither  &  $\bar{~}\bar{~}|\underline{~~}$ & $P_{2}$ \\ \hline
    3   &  Light   &  Left     &  Heavy   &  Neither  &  $\bar{~}\bar{~}|\underline{~~}$ & $P_{2}$ \\ \hline
    4   &  Light   &  Neither  &  Light   &  Left     &  $\bar{~}\bar{~}|\underline{~~}$ & $P_{2}$ \\ \hline
    5   &  Heavy   &  Neither  &  Light   &  Left     &  $\bar{~}\bar{~}|\underline{~~}$ & $P_{2}$ \\ \hline
    6   &  Heavy   &  Right    &  Heavy   &  Right    &  $\bar{~}\bar{~}|\underline{~~}$ & $P_{1}$ \\ \hline
    7   &  Heavy   &  Right    &  Light   &  Neither  &  $\bar{~}\bar{~}|\underline{~~}$ & $P_{2}$ \\ \hline
    8   &  Heavy   &  Right    &  Heavy   &  Neither  &  $\bar{~}\bar{~}|\underline{~~}$ & $P_{2}$ \\ \hline
    9   &  Light   &  Neither  &  Heavy   &  Right    &  $\bar{~}\bar{~}|\underline{~~}$ & $P_{2}$ \\ \hline
    10  &  Heavy   &  Neither  &  Heavy   &  Right    &  $\bar{~}\bar{~}|\underline{~~}$ & $P_{2}$ \\ \hline
    11  &  Light   &  Right    &  Heavy   &  Left     &  $\bar{~}\bar{~}|\underline{~~}$ & $P_{2}$ \\ \hline
    12  &  Heavy   &  Left     &  Light   &  Left     &  $\bar{~}\bar{~}|\underline{~~}$ & $P_{1}$ \\ \hline\hline

    13  &  Light   &  Right    &  Light   &  Right    &  $\underline{~~}|\bar{~}\bar{~}$ & $P_{1}$ \\ \hline
    14  &  Light   &  Right    &  Light   &  Neither  &  $\underline{~~}|\bar{~}\bar{~}$ & $P_{2}$ \\ \hline
    15  &  Light   &  Right    &  Heavy   &  Neither  &  $\underline{~~}|\bar{~}\bar{~}$ & $P_{2}$ \\ \hline
    16  &  Light   &  Neither  &  Light   &  Right    &  $\underline{~~}|\bar{~}\bar{~}$ & $P_{2}$ \\ \hline
    17  &  Heavy   &  Neither  &  Light   &  Right    &  $\underline{~~}|\bar{~}\bar{~}$ & $P_{2}$ \\ \hline
    18  &  Heavy   &  Left     &  Heavy   &  Left     &  $\underline{~~}|\bar{~}\bar{~}$ & $P_{1}$ \\ \hline
    19  &  Heavy   &  Left     &  Light   &  Neither  &  $\underline{~~}|\bar{~}\bar{~}$ & $P_{2}$ \\ \hline
    20  &  Heavy   &  Left     &  Heavy   &  Neither  &  $\underline{~~}|\bar{~}\bar{~}$ & $P_{2}$ \\ \hline
    21  &  Light   &  Neither  &  Heavy   &  Left     &  $\underline{~~}|\bar{~}\bar{~}$ & $P_{2}$ \\ \hline
    22  &  Heavy   &  Neither  &  Heavy   &  Left     &  $\underline{~~}|\bar{~}\bar{~}$ & $P_{2}$ \\ \hline
    23  &  Light   &  Left     &  Heavy   &  Right    &  $\underline{~~}|\bar{~}\bar{~}$ & $P_{2}$  \\ \hline
    34  &  Heavy   &  Right    &  Light   &  Right    &  $\underline{~~}|\bar{~}\bar{~}$ & $P_{1}$ \\ \hline\hline

    25  &  Light   &  Right    &  Light   &  Left     &  $-|-$ & $P_{2}$ \\ \hline
    26  &  Light   &  Left     &  Light   &  Right    &  $-|-$ & $P_{2}$ \\ \hline
    27  &  Heavy   &  Right    &  Heavy   &  Left     &  $-|-$ & $P_{2}$ \\ \hline
    28  &  Heavy   &  Left     &  Heavy   &  Right    &  $-|-$ & $P_{2}$ \\ \hline
    29  &  Light   &  Neither  &  Light   &  Neither  &  $-|-$ & $P_{1}$ \\ \hline
    30  &  Light   &  Neither  &  Heavy   &  Neither  &  $-|-$ & $P_{1}$ \\ \hline
    31  &  Heavy   &  Neither  &  Light   &  Neither  &  $-|-$ & $P_{1}$ \\ \hline
    32  &  Heavy   &  Neither  &  Heavy   &  Neither  &  $-|-$ & $P_{1}$ \\ \hline\hline

    33  &  Light   &  Left     &  Heavy   &  Left    &  \textbf{Unknown} & $P_{1}$ \\ \hline
    34  &  Heavy   &  Left     &  Right   &  Left    &  \textbf{Unknown} & $P_{1}$ \\ \hline
    35  &  Light   &  Right    &  Heavy   &  Right   &  \textbf{Unknown} & $P_{1}$ \\ \hline
    36  &  Heavy   &  Right    &  Light   &  Right   &  \textbf{Unknown} & $P_{1}$ \\ \hline
  \end{tabular}
\end{table}

  The column labeled ``Probability'' in Table~\ref{tab:p1TwoBallsUnknownWeights} denotes the likelihood of the corresponding balance configuration.  $P_{1}$~and $P_{2}$ equal~$\frac{3}{154}$ and~$\frac{4}{121}$, respectively.\footnote{These values are based on four balls each in the left and right balance as well as four off to the side.}  Table~\ref{tab:twoBallProbabilityBreakdown} lists the grouped probability of each balance outcome.  Using this table to calculate binary entropy is problematic as the ``Unknown'' cases do not fit neatly into any three of the balance outcomes.  For simplicity, we spread the probability for the ``Unknown'' cases equally across the three outcomes.  Therefore, this \textit{approximated entropy} equals~$1.552$.   Using Eq.~\eref{eq:p1NumbWeightsRequired}, we estimate the expected number of weights required for two odd balls is:

  \[ \ceil*{\frac{\log_{2} 264}{H_{approx}}} \approx \ceil*{\frac{8.044}{1.552}} \approx \ceil*{5.182} = \boxed{6} \text{.} \]

  When there are three odd balls, $\abs{\mathcal{H}}$ equals~${2\cdot2\cdot2\cdot\binom{12}{3} = 1,760}$.  Estimating the entropy is made more challenging than two odd balls as the probability of the ``Unknown'' configurations rises.  To simplify the calculation, we reuse the approximated entropy from the two ball case.  This results in an estimate of

  \[ \ceil*{\frac{\log_{2} 1760}{H_{approx}}} \approx \ceil*{\frac{10.78}{1.552}} \approx \ceil*{6.945} = \boxed{7}\]

  \noindent
  for the expected number of weights.

  \begin{table}
    \centering
    \caption{Probability partition by balance outcome for two odd balls of unknown relative weights}\label{tab:twoBallProbabilityBreakdown}
    \begin{tabular}{|c||c|c|c|c|}
      \hline
      Balance Position & $\bar{~}\bar{~}|\underline{~~}$  & $\underline{~~}|\bar{~}\bar{~}$  & $-|-$ & Unknown \\\hline
      Probability      & $\frac{603}{1694} \approx 0.356$ & $\frac{603}{1694} \approx 0.356$ & $\frac{178}{847} \approx 0.210$ & $\frac{12}{154} \approx 0.078$    \\\hline
    \end{tabular}
  \end{table}

  To estimate the worst case number of weights, Eq.~\eref{eq:p1NumbWeightsRequired} changes slightly as shown below.

  \begin{equation}\label{eq:p1NumbWeightsWorstCase}
    O\left[\#\text{Weights}\right] = \ceil*{\frac{\log_{2} \abs{\mathcal{H}}}{-\log_{2} \left(\max \{\Pr(\text{Balance})\} \right) }}
  \end{equation}

  Using Table~\ref{tab:twoBallProbabilityBreakdown}, the $\Pr(\text{Balance})$ equals~0.356.  Accounting for the equal spreading of the ``Unknown'' probability, we will use probability~0.382 in the calculations below.  Therefore, the worst case number of weighings for two balls is estimated as:

  \[ O\left[\#\text{Weights Two Balls}\right] \approx \ceil*{\frac{\log_{2} 264}{-\log_{2} 0.382}} \approx \ceil{5.792} = \boxed{6} \text{.} \]

  \noindent
  Similarly, for three balls, the upper bound is:

  \[ O\left[\#\text{Weights Three Balls}\right] \approx \ceil*{\frac{\log_{2} 1760}{-\log_{2} 0.382}} \approx \ceil{7.766} = \boxed{8} \text{.} \]

  \noindent
  \textit{Conclusion}:

  For two odd balls, both the expected and upper bound calculations estimated that 6~weights are required.  This provides very high confidence in the estimation.  When there are three odd balls, the upper bound of 8~weights appears to be a more appropriate estimation.

\begin{subproblem}
  How do your answers change if it is known that all the regular balls weigh 100g, that light balls weight 99g, and heavy ones weigh 110g?
\end{subproblem}

  In part~(a), uncertainty about the balance's behavior in some cases necessitated that we approximate the binary entropy.  Now that the relative ball weights are known, Cases~\#33-34 and Cases~\#35-36 in Table~\ref{tab:p1TwoBallsUnknownWeights} are assigned to ``${\underline{~~}|\bar{~}\bar{~}}$'' and ``${\bar{~}\bar{~}|\underline{~~}}$'' respectively.  The updated probability breakdown for each balance outcome is shown in Table~\ref{tab:twoBallProbabilityBreakdownKnownWeights}.  This corresponds to a binary entropy of~$1.531$.  Note that even though we have more information, the binary entropy went down.  That is because when approximating the entropy in part~(a), we spread the probability corresponding to the ``Unknown'' case evenly across the three balance outcomes.  This lead to a more balanced probability mass function (pmf), which in turn led to a higher entropy.

  \begin{table}
    \centering
    \caption{Updated probability partition with known ball weights in part~(b)}\label{tab:twoBallProbabilityBreakdownKnownWeights}
    \begin{tabular}{|c||c|c|c|}
      \hline
      Balance Position & $\bar{~}\bar{~}|\underline{~~}$  & $\underline{~~}|\bar{~}\bar{~}$  & $-|-$ \\\hline
      Probability      & $\frac{669}{1694} \approx 0.395$ & $\frac{669}{1694} \approx 0.395$ & $\frac{178}{847} \approx 0.210$   \\\hline
    \end{tabular}
  \end{table}

  Using the updated entropy, the expected number of weights for two balls remains~6, but the worst case was $\ceil{6.003} = 7$.  Given that the worst case is only marginally greater than six before applying the ceiling function, we hypothesize that six measurements is the better estimate.

  For three balls, the expected number of weights increased to~${\ceil*{7.042} = 8}$ and the worst case estimate was ${\ceil{8.045} = 9}$.  Both the expected and worst case estimates are only slightly more than their previous estimate.  We estimate that splitting the difference and estimating $8$~weights is the best estimate for three balls.