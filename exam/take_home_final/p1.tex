\begin{problem}{4}{14}
  You are given 12 balls and the three-outcome balance of exercise 4.1; this time, two of the balls are odd; each odd ball may be heavy or light, and we don't know which.  We want to identify the odd balls and in which direction they are odd.
\end{problem}

\begin{subproblem}
  \textit{Estimate} how many weights are required by the optimal strategy.  And what if there are three odd balls?
\end{subproblem}

  From a set of 12 balls, there are $\binom{12}{2}$ combinations of two ``odd'' balls.  Similarly, each ``odd'' ball can be either light or heavy.  Therefore, the number of unique solutions is:

  \[ 2\cdot 2 \cdot \binom{12}{2} = 4 \cdot 66 = 264 \text{,} \]

  all of which are equally likely.  Ideally, the three-outcome balance roughly reduces the set of possible solutions by two-thirds with each measurement.  A best case expected number of weights for two odd balls is:

  \[ \log_{3} 264\approx 5.075 {.} \]

  Since 5.075 is much closer to~5 than~6, there is still slack for imbalance in the probabilities of the respective outcomes of a ball weighing.  Therefore, I estimate \boxed{6}~weights are required for two odd balls.

  When there are three odd balls, then there are $\binom{12}{3}$ combinations of ``odd'' balls.  Again, since each odd ball is either light or heavy, the number of unique (equally likely) solutions is:

  \[ 2^3 \cdot \binom{12}{3} = 8 \cdot 220 = 1760 \text{.} \]

  As such, the minimum expected number of weightings for three odd balls is:

  \[ \log_{3} 1760 \approx 6.802 \text{.} \]

  Unfortunately, $6.802$ is close to $7$ leaving less slack for any imbalance in the outcome probabilities.  For that reason, I estimate the expected number of weights for three balls is \boxed{7 \text{ or } 8}.

\begin{subproblem}
  How do your answers change if it is known that all the regular balls weigh 100g, that light balls weight 99g, and heavy ones weigh 110g?
\end{subproblem}


