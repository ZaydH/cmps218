\newpage
\begin{problem}{22}{5}{15}
  The seven scientists. $N$~datapoints $\{x_n\}$ are drawn from $N$~distributions, all of which are Gaussian with a common mean~$\mu$ but with different unknown standard deviations~$\sigma_{n}$.  What are the maximum likelihood parameter $\mu$,$\{\sigma_{n}\}$ given the data?  For example, seven scientists (A, B, C, D, E, F, G) with wildly-differing experimental skills to measure~$\mu$.  You expect some of them to do accurate work (i.e.,~to have small~$\sigma_n$), and some of them to turn in wildly inaccurate results (i.e.,~to have enormous~$\sigma_n$).  Table~\ref{tab:problem22.5.15} shows their seven results.  What is the~$\mu$, and how reliable is each scientist?

  \begin{table}[h]
    \centering
    \begin{tabular}{c|S}
      \hline
      Scientist & $x_n$     \\\hline
      A         & -27.020   \\\hline
      B         & 3.570     \\\hline
      C         & 8.191     \\\hline
      D         & 9.898     \\\hline
      E         & 9.603     \\\hline
      F         & 9.945     \\\hline
      G         & 10.056    \\\hline
    \end{tabular}
    \caption{Seven measurements $\{x_n\}$ of a parameter $\mu$ by seven scientists each having his own noise-level $\sigma_n$.}\label{tab:problem22.5.15}
  \end{table}

  I hope that you agree that, intuitively, it looks pretty certain that A and B are both inept measurers, that D-G are better, and that the true value $\mu$ is somewhere close to~10.  But what does maximizing the likelihood tell you?
\end{problem}

Given $n$ observers that each make a single observation with a common mean $\mu$ and standard deviations $\{\sigma_n\}$, the maximum likelihood mean, $\bar{x}$, is:

\[\bar{x} = \frac{\sum_{i=1}^{n}x_i}{n}\textrm{.}\]

\noindent
In the case of the seven scientists, the sample mean is:

\begin{aligncustom}
  \bar{x} &= \frac{-27.020 + 3.570 + 8.191 + 9.898 + 9.603 + 9.945 + 10.056}{7}\\
          &= \boxed{3.463}
\end{aligncustom}

For observer $i$ where ${1 \leq i \leq n}$, the maximum likelihood standard deviation is:

\[ \sigma_i = \lvert x_i - \mu \rvert \]

\noindent
since there is only one sample per distribution. Table~\ref{tab:problem22.5.15Stdev} lists the maximum likelihood standard deviations for the seven scientists.

\begin{table}[h]
  \centering
  \begin{tabular}{c|S}
    \hline
    Scientist & $x_n$     \\\hline
    A         & 30.483    \\\hline
    B         & 0.107     \\\hline
    C         & 4.728     \\\hline
    D         & 6.435     \\\hline
    E         & 6.140     \\\hline
    F         & 6.482     \\\hline
    G         & 6.593    \\\hline
  \end{tabular}
  \caption{Maximum likelihood $\sigma$ for the seven scientists}\label{tab:problem22.5.15Stdev}
\end{table}

From the data, it appears that scientists D\textendash G are reliable.  Scientist C appears less reliable than them, but better than A and B which appear to be the worst.

Clearly for this problem, maximizing the likelihood does not yield the most plausible outcome.  There are four observers that essentially measure the same value; it is theoretically possible these measurements are due to coincidence.  However, that it is unlikely given the problem description.  Therefore, relying blindly on the maximum likelihood calculation may lead to poor conclusions.