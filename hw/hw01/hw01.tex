\documentclass{report}

\usepackage{fullpage}
\usepackage[skip=4pt]{caption} % ``skip'' sets the spacing between the figure and the caption.
\usepackage{tikz}
\usepackage{pgfplots}   % Needed for plotting
\pgfplotsset{compat=newest}
\usepgfplotslibrary{fillbetween}  % Allow for highlighting under a curve
\usepackage{amsmath}    % Allows for piecewise functions using the ``cases'' construct
\newcommand{\abs}[1]{\lvert#1\rvert}%
\newcommand{\ceil}[1]{\lceil #1 \rceil}
\newcommand{\floor}[1]{\lfloor #1 \rfloor}
%\usepackage{mathrsfs}   % Use the ``\mathscr'' command in an equation.
\usepackage{siunitx}    % Allows for ``S'' alignment in table to align by decimal point
\usepackage{enumitem}   % Reference enumerated item lists.

\usepackage[obeyspaces,spaces]{url} % Used for typesetting with the ``path'' command
\usepackage[hidelinks]{hyperref}   % Make the cross references clickable hyperlinks
\usepackage[bottom]{footmisc} % Prevents the table going below the footnote
\usepackage{nccmath}    % Needed in the workaround for the ``aligncustom'' environment
\usepackage{amssymb}    % Used for black QED symbol
\usepackage{bm}    % Allows for bolding math symbols.
\usepackage{tabto}     % Allows to tab to certain point on a line
\usepackage{float}

\newcommand{\hangindentdistance}{1cm}
\setlength{\parindent}{0pt}
\setlength{\leftskip}{\hangindentdistance}
\setlength{\hangafter}{1}
\setlength{\parskip}{1em}


% Set up page formatting
\usepackage{todonotes}
\usepackage{fancyhdr} % Used for every page footer and title.
\pagestyle{fancy}
\fancyhf{} % Clears both the header and footer
\renewcommand{\headrulewidth}{0pt} % Eliminates line at the top of the page.
\fancyfoot[LO]{CMPS218 \textendash\ Homework \#1} % Left
\fancyfoot[CO]{\thepage} % Center
\fancyfoot[RO]{Zayd Hammoudeh} %Right

% Change interline spacing.
\renewcommand{\baselinestretch}{1.1}
\newenvironment{aligncustom}
{ \csname align*\endcsname % Need to do this instead of \begin{align*} because of LaTeX bug.
    \centering
}
{
  \csname endalign*\endcsname
}
%--------------------------------------------------


\title{\textbf{CMPS218 \textendash\ Homework \#1}}
\author{Zayd Hammoudeh}

%---------------------------------------------------%
% Define the Environments for the Problem Inclusion %
%---------------------------------------------------%
\newcounter{subProbCount}       % Initialize the subproblem counter
\newcounter{problemCount}
\setcounter{problemCount}{0} % Reset the subproblem counter
\newenvironment{problemshell}{
  \par%
  \medskip
  \leftskip=0pt\rightskip=0pt%
}
{
  \par\medskip
  \setcounter{subProbCount}{1} % Reset the subproblem counter
}
\newenvironment{problem}[3]
{%
  \begin{problemshell}
    \noindent \textit{Chapter \##1.#2, Problem \##3} \\
    \bfseries
}
{
  \end{problemshell}
}
\newenvironment{subproblem}
{%
  \par%
  \medskip
  \leftskip=0pt\rightskip=0pt%
  \bfseries
  % Print the subproblem count and offset to the left
  (\alph{subProbCount}) \hangindent=\hangindentdistance \hangafter=1 \tabto{\hangindentdistance}
}
{
  \stepcounter{subProbCount} % Increment the subproblem counter
  \par\medskip
}
\newcommand{\problemspace}{\\[0.4em]}

\begin{document}
  \maketitle

  \textbf{Name}: Zayd Hammoudeh

  \textbf{Assignment}:  CMPS218 Homework~\#1

  \textbf{Other Student Discussions}: I discussed the problems in this homework with the following students:
  \vspace{-3em}
  \begin{itemize}[leftmargin=1.35cm]
    \setlength\itemsep{0em}
    \item Hadley Black -- He asked me about the first problem.

    \item Noujan Pashanasangi -- We discussed second problem concerning the poisoned glass and the third problem concerning soft K-Means stiffness.

    \item Will Bolden -- We discussed the second and third problems.  In particular, we discussed the fourth problem in depth.

    \item Konstantinos Zampetakis and Keller Jordan -- We discussed the fourth problem.
  \end{itemize}

  \textbf{Problem Assessments}: All problems are complete to the best of my understanding.  The answer to the fourth problem was the most challenging and is the one where I believe my answer may be the most suspect.

  \begin{problem}{4}{14}
  You are given 12 balls and the three-outcome balance of exercise 4.1; this time, two of the balls are odd; each odd ball may be heavy or light, and we don't know which.  We want to identify the odd balls and in which direction they are odd.
\end{problem}

  \noindent
  \textit{Assumptions}:

  \begin{itemize}
    \item The problem states, ``each odd ball may be heavy or light.'' This language implies that all heavy odd balls have equivalent weight.  Similarly, the weights of all light odd balls are equivalent.
  \end{itemize}

  \noindent
  \textit{Answer}:

  The hypothesis class,~$\mathcal{H}$, is the set of all valid allocations of odd ball(s) in the problem. Given a three-outcome balance, the \textit{lower bound} for the required number of weights is:

  \begin{equation}
    \Omega\left[\#\text{Weights}\right] = \ceil*{\log_{3} \abs{\mathcal{H}}} \text{.}
  \end{equation}

  \noindent
  The \textit{expected} number of weights can be estimated via:

  \begin{equation}\label{eq:p1NumbWeightsRequired}
    \mathbb{E}\left[\#\text{Weights}\right] \approx \ceil*{\frac{\log_{2} \abs{\mathcal{H}}}{H}} \text{,}
  \end{equation}

  \noindent
  where $H$ is the binary entropy of the balance outcomes.  Eq.~\eref{eq:p1NumbWeightsRequired}'s correctness is quite intuitive. Representing each member of~$\mathcal{H}$ requires ${\log_{2}\abs{\mathcal{H}}}$~bits.  Binary entropy is the \textit{expected} information gain per weighing (in bits).  The ratio of the two in turn yields the \textit{expected number of weights}.

  The quality of the estimate Eq.~\eref{eq:p1NumbWeightsRequired} provides is built on two assumptions.  First, since entropy~$H$ is used as a normalizer, the equation assumes that~$H$ remains constant in each weighing.  Next, if the probability distribution associated with the balance outcomes is far from uniform, then there will be a large divergence between this expected quantity and the upper bound (which is more relevant for Mackay's question).

  To verify the Eq.~\eref{eq:p1NumbWeightsRequired}'s applicability, consider the case of a single odd ball.  The cardinality of~$\mathcal{H}$ is ${2 * \binom{12}{1} = 24}$. Each outcome of the balance partitions~$\mathcal{H}$ into subsets of roughly equal probability. This is illustrated in Table~\ref{tab:p1OneBallTable} which shows the correspondence between the odd ball's placement and the balance outcome. Using Eq.~\eref{eq:p1NumbWeightsRequired}, we estimate the expected number of weights required is approximately

  \[ \ceil*{\frac{\log_{2}24}{1.585}} \approx \ceil*{\frac{4.584}{1.585}} \approx \ceil{2.893} = 3 \text{.} \]

  \noindent
  This estimate equals the actual number of weights as shown in Figure~4.2 of Mackay's text.

  \begin{table}[H]
    \centering
    \caption{Three-outcome balance position for a \\single odd ball that is light or heavy}\label{tab:p1OneBallTable}
    \begin{tabular}{|c||c|c||c|}
      \hline
      \multirow{2}{*}{Case~\#} &  \multicolumn{2}{c||}{Odd Ball} & Balance\\\cline{2-3}
      &  Weight  &  Location &  Position  \\ \hline\hline
      1   &  Light   &  Left     &  $\bar{~}\bar{~}|\underline{~~}$ \\ \hline
      2   &  Heavy   &  Right    &  $\bar{~}\bar{~}|\underline{~~}$ \\ \hline\hline
      3   &  Light   &  Right    &  $\underline{~~}|\bar{~}\bar{~}$ \\ \hline
      4   &  Heavy   &  Left     &  $\underline{~~}|\bar{~}\bar{~}$ \\ \hline\hline
      5   &  Light   &  Neither  &  $-|-$ \\ \hline
      6   &  Heavy   &  Neither  &  $-|-$ \\ \hline
    \end{tabular}
  \end{table}


\begin{subproblem}
  \textit{Estimate} how many weights are required by the optimal strategy.  And what if there are three odd balls?
\end{subproblem}

  When there are two odd balls, $\abs{\mathcal{H}}$ is ${2\cdot2\cdot\binom{12}{2} = 264}$. Table~\ref{tab:p1TwoBallsUnknownWeights} shows the possible balance locations for two balls.  Since this part of the problem provides no information about the relative ball weights, it is not possible to know \textit{a priori} the balance's behavior when there are mismatched balls on the same side of the balance.

  \begin{table}
  \centering
  \caption{Three-outcome balance position for two\\odd balls that are light and/or heavy Ball}\label{tab:p1TwoBallsUnknownWeights}
  \begin{tabular}{|c||c|c||c|c||c||c|}
    \hline
    \multirow{2}{*}{Case~\#} & \multicolumn{2}{c||}{Odd Ball~\#1} &  \multicolumn{2}{c||}{Odd Ball~\#2} & Balance & \multirow{2}{*}{$\Pr$}\\\cline{2-5}
    &  Weight  &  Location &  Weight  &  Location &  Position                        &         \\ \hline\hline
    1   &  Light   &  Left     &  Light   &  Left     &  $\bar{~}\bar{~}|\underline{~~}$ & $P_{1}$ \\ \hline
    2   &  Light   &  Left     &  Light   &  Neither  &  $\bar{~}\bar{~}|\underline{~~}$ & $P_{2}$ \\ \hline
    3   &  Light   &  Left     &  Heavy   &  Neither  &  $\bar{~}\bar{~}|\underline{~~}$ & $P_{2}$ \\ \hline
    4   &  Light   &  Neither  &  Light   &  Left     &  $\bar{~}\bar{~}|\underline{~~}$ & $P_{2}$ \\ \hline
    5   &  Heavy   &  Neither  &  Light   &  Left     &  $\bar{~}\bar{~}|\underline{~~}$ & $P_{2}$ \\ \hline
    6   &  Heavy   &  Right    &  Heavy   &  Right    &  $\bar{~}\bar{~}|\underline{~~}$ & $P_{1}$ \\ \hline
    7   &  Heavy   &  Right    &  Light   &  Neither  &  $\bar{~}\bar{~}|\underline{~~}$ & $P_{2}$ \\ \hline
    8   &  Heavy   &  Right    &  Heavy   &  Neither  &  $\bar{~}\bar{~}|\underline{~~}$ & $P_{2}$ \\ \hline
    9   &  Light   &  Neither  &  Heavy   &  Right    &  $\bar{~}\bar{~}|\underline{~~}$ & $P_{2}$ \\ \hline
    10  &  Heavy   &  Neither  &  Heavy   &  Right    &  $\bar{~}\bar{~}|\underline{~~}$ & $P_{2}$ \\ \hline
    11  &  Light   &  Right    &  Heavy   &  Left     &  $\bar{~}\bar{~}|\underline{~~}$ & $P_{2}$ \\ \hline
    12  &  Heavy   &  Left     &  Light   &  Left     &  $\bar{~}\bar{~}|\underline{~~}$ & $P_{1}$ \\ \hline\hline

    13  &  Light   &  Right    &  Light   &  Right    &  $\underline{~~}|\bar{~}\bar{~}$ & $P_{1}$ \\ \hline
    14  &  Light   &  Right    &  Light   &  Neither  &  $\underline{~~}|\bar{~}\bar{~}$ & $P_{2}$ \\ \hline
    15  &  Light   &  Right    &  Heavy   &  Neither  &  $\underline{~~}|\bar{~}\bar{~}$ & $P_{2}$ \\ \hline
    16  &  Light   &  Neither  &  Light   &  Right    &  $\underline{~~}|\bar{~}\bar{~}$ & $P_{2}$ \\ \hline
    17  &  Heavy   &  Neither  &  Light   &  Right    &  $\underline{~~}|\bar{~}\bar{~}$ & $P_{2}$ \\ \hline
    18  &  Heavy   &  Left     &  Heavy   &  Left     &  $\underline{~~}|\bar{~}\bar{~}$ & $P_{1}$ \\ \hline
    19  &  Heavy   &  Left     &  Light   &  Neither  &  $\underline{~~}|\bar{~}\bar{~}$ & $P_{2}$ \\ \hline
    20  &  Heavy   &  Left     &  Heavy   &  Neither  &  $\underline{~~}|\bar{~}\bar{~}$ & $P_{2}$ \\ \hline
    21  &  Light   &  Neither  &  Heavy   &  Left     &  $\underline{~~}|\bar{~}\bar{~}$ & $P_{2}$ \\ \hline
    22  &  Heavy   &  Neither  &  Heavy   &  Left     &  $\underline{~~}|\bar{~}\bar{~}$ & $P_{2}$ \\ \hline
    23  &  Light   &  Left     &  Heavy   &  Right    &  $\underline{~~}|\bar{~}\bar{~}$ & $P_{2}$  \\ \hline
    34  &  Heavy   &  Right    &  Light   &  Right    &  $\underline{~~}|\bar{~}\bar{~}$ & $P_{1}$ \\ \hline\hline

    25  &  Light   &  Right    &  Light   &  Left     &  $-|-$ & $P_{2}$ \\ \hline
    26  &  Light   &  Left     &  Light   &  Right    &  $-|-$ & $P_{2}$ \\ \hline
    27  &  Heavy   &  Right    &  Heavy   &  Left     &  $-|-$ & $P_{2}$ \\ \hline
    28  &  Heavy   &  Left     &  Heavy   &  Right    &  $-|-$ & $P_{2}$ \\ \hline
    29  &  Light   &  Neither  &  Light   &  Neither  &  $-|-$ & $P_{1}$ \\ \hline
    30  &  Light   &  Neither  &  Heavy   &  Neither  &  $-|-$ & $P_{1}$ \\ \hline
    31  &  Heavy   &  Neither  &  Light   &  Neither  &  $-|-$ & $P_{1}$ \\ \hline
    32  &  Heavy   &  Neither  &  Heavy   &  Neither  &  $-|-$ & $P_{1}$ \\ \hline\hline

    33  &  Light   &  Left     &  Heavy   &  Left    &  \textbf{Unknown} & $P_{1}$ \\ \hline
    34  &  Heavy   &  Left     &  Right   &  Left    &  \textbf{Unknown} & $P_{1}$ \\ \hline
    35  &  Light   &  Right    &  Heavy   &  Right   &  \textbf{Unknown} & $P_{1}$ \\ \hline
    36  &  Heavy   &  Right    &  Light   &  Right   &  \textbf{Unknown} & $P_{1}$ \\ \hline
  \end{tabular}
\end{table}

  The column labeled ``Probability'' in Table~\ref{tab:p1TwoBallsUnknownWeights} denotes the likelihood of the corresponding balance configuration.  $P_{1}$~and $P_{2}$ equal~$\frac{3}{154}$ and~$\frac{4}{121}$, respectively.\footnote{These values are based on four balls each in the left and right balance as well as four off to the side.}  Table~\ref{tab:twoBallProbabilityBreakdown} lists the grouped probability of each balance outcome.  Using this table to calculate binary entropy is problematic as the ``Unknown'' cases do not fit neatly into any three of the balance outcomes.  For simplicity, we spread the probability for the ``Unknown'' cases equally across the three outcomes.  Therefore, this \textit{approximated entropy} equals~$1.552$.   Using Eq.~\eref{eq:p1NumbWeightsRequired}, we estimate the expected number of weights required for two odd balls is:

  \[ \ceil*{\frac{\log_{2} 264}{H_{approx}}} \approx \ceil*{\frac{8.044}{1.552}} \approx \ceil*{5.182} = \boxed{6} \text{.} \]

  When there are three odd balls, $\abs{\mathcal{H}}$ equals~${2\cdot2\cdot2\cdot\binom{12}{3} = 1,760}$.  Estimating the entropy is made more challenging than two odd balls as the probability of the ``Unknown'' configurations rises.  To simplify the calculation, we reuse the approximated entropy from the two ball case.  This results in an estimate of

  \[ \ceil*{\frac{\log_{2} 1760}{H_{approx}}} \approx \ceil*{\frac{10.78}{1.552}} \approx \ceil*{6.945} = \boxed{7}\]

  \noindent
  for the expected number of weights.

  \begin{table}
    \centering
    \caption{Probability partition by balance outcome for two odd balls of unknown relative weights}\label{tab:twoBallProbabilityBreakdown}
    \begin{tabular}{|c||c|c|c|c|}
      \hline
      Balance Position & $\bar{~}\bar{~}|\underline{~~}$  & $\underline{~~}|\bar{~}\bar{~}$  & $-|-$ & Unknown \\\hline
      Probability      & $\frac{603}{1694} \approx 0.356$ & $\frac{603}{1694} \approx 0.356$ & $\frac{178}{847} \approx 0.210$ & $\frac{12}{154} \approx 0.078$    \\\hline
    \end{tabular}
  \end{table}

  To estimate the worst case number of weights, Eq.~\eref{eq:p1NumbWeightsRequired} changes slightly as shown below.

  \begin{equation}\label{eq:p1NumbWeightsWorstCase}
    O\left[\#\text{Weights}\right] = \ceil*{\frac{\log_{2} \abs{\mathcal{H}}}{-\log_{2} \left(\max \{\Pr(\text{Balance})\} \right) }}
  \end{equation}

  Using Table~\ref{tab:twoBallProbabilityBreakdown}, the $\Pr(\text{Balance})$ equals~0.356.  Accounting for the equal spreading of the ``Unknown'' probability, we will use probability~0.382 in the calculations below.  Therefore, the worst case number of weighings for two balls is estimated as:

  \[ O\left[\#\text{Weights Two Balls}\right] \approx \ceil*{\frac{\log_{2} 264}{-\log_{2} 0.382}} \approx \ceil{5.792} = \boxed{6} \text{.} \]

  \noindent
  Similarly, for three balls, the upper bound is:

  \[ O\left[\#\text{Weights Three Balls}\right] \approx \ceil*{\frac{\log_{2} 1760}{-\log_{2} 0.382}} \approx \ceil{7.766} = \boxed{8} \text{.} \]

  \noindent
  \textit{Conclusion}:

  For two odd balls, both the expected and upper bound calculations estimated that 6~weights are required.  This provides very high confidence in the estimation.  When there are three odd balls, the upper bound of 8~weights appears to be a more appropriate estimation.

\begin{subproblem}
  How do your answers change if it is known that all the regular balls weigh 100g, that light balls weight 99g, and heavy ones weigh 110g?
\end{subproblem}

  In part~(a), uncertainty about the balance's behavior in some cases necessitated that we approximate the binary entropy.  Now that the relative ball weights are known, Cases~\#33-34 and Cases~\#35-36 in Table~\ref{tab:p1TwoBallsUnknownWeights} are assigned to ``${\underline{~~}|\bar{~}\bar{~}}$'' and ``${\bar{~}\bar{~}|\underline{~~}}$'' respectively.  The updated probability breakdown for each balance outcome is shown in Table~\ref{tab:twoBallProbabilityBreakdownKnownWeights}.  This corresponds to a binary entropy of~$1.531$.  Note that even though we have more information, the binary entropy went down.  That is because when approximating the entropy in part~(a), we spread the probability corresponding to the ``Unknown'' case evenly across the three balance outcomes.  This lead to a more balanced probability mass function (pmf), which in turn led to a higher entropy.

  \begin{table}
    \centering
    \caption{Updated probability partition with known ball weights in part~(b)}\label{tab:twoBallProbabilityBreakdownKnownWeights}
    \begin{tabular}{|c||c|c|c|}
      \hline
      Balance Position & $\bar{~}\bar{~}|\underline{~~}$  & $\underline{~~}|\bar{~}\bar{~}$  & $-|-$ \\\hline
      Probability      & $\frac{669}{1694} \approx 0.395$ & $\frac{669}{1694} \approx 0.395$ & $\frac{178}{847} \approx 0.210$   \\\hline
    \end{tabular}
  \end{table}

  Using the updated entropy, the expected number of weights for two balls remains~6, but the worst case was $\ceil{6.003} = 7$.  Given that the worst case is only marginally greater than six before applying the ceiling function, we hypothesize that six measurements is the better estimate.

  For three balls, the expected number of weights increased to~${\ceil*{7.042} = 8}$ and the worst case estimate was ${\ceil{8.045} = 9}$.  Both the expected and worst case estimates are only slightly more than their previous estimate.  We estimate that splitting the difference and estimating $8$~weights is the best estimate for three balls.

  \newpage
\begin{problem}{5}{8}{30}
  \textit{Scientific American} carried the following puzzle in 1975.
  \problemspace
  \textbf{The poisoned glass:} \textnormal{\textit{'Mathematicians are curious birds,' the police commissioner said to his wife. 'You see, we had all those partly filled glasses lined up in rows on a table in the hotel kitchen. Only one contained poison, and we wanted to know which one before searching the glass for fingerprints.  Our lab could test the liquid in each glass, but the tests take time and money, so we wanted to make as few of them as possible by simultaneously testing mixtures of small samples from groups of glasses.  The university sent over a mathematics professor to help us.  He counted the glasses, smiled and said: \\ ``Pick any glass you want, Commissioner. We'll test it first.'' \\ ``But won't that waste a test?'' I asked. \\ ``No,'' he said. ``it's part of the best procedure.  We can test one glass first.  It doesn't matter which one.'' \\ `How many glasses were there to start with?' the commissioner's wife asked. \\ 'I don't remember. Somewhere between 100 and 200.'} \\What was the exact number of glasses?}
  \problemspace
  Solve this puzzle and then explain why the professor is in fact wrong and the commissioner was right.  What is in fact the optimal procedure for identifying the one poisoned glass?  What is the expected waste relative to this optimum if one followed the professor's strategy?  Explain the relationship to symbol coding.
\end{problem}

The test for poison has a binary outcome, i.e.,~the sample either has poison or not.  Therefore, assuming each cup has poison with equal probability, the size of the remaining set of glasses is, on average, cut in half with each test.

If the number of glasses,~$n$, is a power of~$2$, then the number of tests required is $\lg n$, where $\lg$ is the base-$2$ logarithm.  Note that the only power of~2 between 100 and 200 is 128.  There was one extra glass that the professor tested separately.  Therefore, there was \boxed{129~\text{glasses}}.

\begin{table}[h]
  \centering
  \begin{tabular}{c|c|c}
    \hline
    Glass ID & Probability of Poison & \# Tests  \\\hline
    1        & 1/129                 & 1         \\\hline
    2-129    & 128/129               & 1 + 7 = 8 \\\hline
  \end{tabular}
  \caption{Number of tests required using the professor's strategy}\label{tab:problem5.9.20-Prof}
\end{table}

Table~\ref{tab:problem5.9.20-Prof} shows the number of tests required when using the professor's strategy.  Glass~\#1 represents the first glass tested, i.e.,~the one selected at random.  In the unlikely event that glass has the poison, no additional testing is required. In contrast, if the poison is in one of the other 128~glasses, seven tests (plus the additional one for the first glass) are required.  Using this strategy, the expected number of tests is:

\begin{aligncustom}
  \mathbb{E}(\text{Professor's Strategy}) &= \frac{1}{129} \cdot 1 + \frac{128}{129} \cdot 8 \\
                                          &\approx \boxed{7.946}\text{.}
\end{aligncustom}


In contrast, the optimal strategy is:

\begin{enumerate}
  \item Select one glass at random and leave it off to the side.
  \item\label{itm:test} Test a sample that combines wine from half of the remaining glasses (excluding the one off to the side).
  \item\label{itm:discard} If poison is observed in this tested sample, discard the untested glasses.  Otherwise, discard the tested glasses.
  \item Repeat steps~\#\ref{itm:test} and~\#\ref{itm:discard} until only a single glass remains (excluding the one off to the side).
  \item If poison was ever observed in any of the previous tests, then the remaining glass not off to the side has the poison, and no additional testing is required.
  \item\label{itm:testRemaining} If poison was never observed in any test, then test just the remaining glass not off to the side. If poison is detected, then the answer is clear, and the tested glass has poison; otherwise, the glass off to the side has the poison.
\end{enumerate}

Table~\ref{tab:problem5.9.20-Opt} shows the number of tests required using this optimum strategy.    Note that Glass\#~129 entails the remaining glass in step~\#\ref{itm:testRemaining} where no sample tests positive for poison up to the last remaining glass not off to the side.

\begin{table}[h]
  \centering
  \begin{tabular}{c|c|c}
    \hline
    Glass ID & Probability of Poison & \# Tests  \\\hline
    1        & 1/129                 & 7 + 1     \\\hline
    2-128    & 127/129               & 7         \\\hline
    129      & 1/129                 & 7 + 1     \\\hline
  \end{tabular}
  \caption{Number of tests required using the optimum strategy}\label{tab:problem5.9.20-Opt}
\end{table}

\noindent
Using this optimum strategy, the expected number of tests is:

\begin{aligncustom}
  \mathbb{E}(\text{Optimum Strategy}) &= \frac{2}{129} \cdot 8 + \frac{127}{129} \cdot 7 \\
  &\approx \boxed{7.016}\text{.}
\end{aligncustom}

It is clear then that the expected waste of the professor's strategy is \boxed{0.93} tests.

Maximum compression of a symbol code is achieved by assigning shorter codes (i.e.,~with less bits) to outcomes with higher probability.  In contrast, the professor prioritized the least likely outcome by testing the randomly selected glass first.  The optimal strategy described above always tests the most likely outcome (i.e.,~more glasses at once) similar to how symbol codes are encoded.

  \newpage
\begin{problem}{15}{5}
  In a magic trick, there are three participants: the magician, an assistant, and a volunteer.  The assistant, who claims to have paranormal abilities, is in a soundproof room.  The magician gives the volunteer six blank cards, five white and one blue.  The volunteer writes a different integer from 1 to 100 on each card, as the magician is watching.  The volunteer keeps the blue card. The magician arranges the white cards in some order and passes them to the assistant. The assistant then announces the number on the blue card.

  How does this trick work?
\end{problem}

For the sake of making this problem meaningful, I am going to \textit{assume} the assistant does not actually have paranormal abilities.

Consider a set of five distinct, arbitrary integers.  Assign these five integers to the letters $a$, $b$, $c$, $d$, and~$e$ satisfying the relationship ${a < b < c < d < e}$.  At this point, the original numbers can be forgotten and only the five letters considered since they will be all that matters.  It is elementary to see that there are ${}_{5}P_{5} = 5! = 120$ permutations of five unique letters.  Assign to each permutation a unique number from 1 to 120.  This represents a uniquely decodable encoding of any number between 1 and 120.

In this trick, the five unique integers written on the white cards are bijectively mapped to the letters $a$ through $e$ as described above.  The magician saw the number written on the blue card and encodes that number based on how he orders (permutes) the five white cards (letters) before passing them to the assistant.  The assistant then decodes this permutation to get the volunteer's number on the blue card completing the trick -- no paranormal abilities needed.



  \begin{problem}{15}{6}
  How does \textit{this} trick work?

  \textnormal{\textit{`Here's an ordinary pack of cards shuffled into random order.  Please choose five cards from the pack, any you wish. Don't let me see their faces. No, don't give them to me: pass them to my assistant Esmerelda.  She can look at them.}}

  \textnormal{\textit{`Now, Esmerelda, show me four of the cards.  Hmm...nine of spades, six of clubs, four of hearts, ten of diamonds. The hidden card must be the queen of spades.}}

  This trick can be performed as described above for a pack of 52 cards.  Use information theory to give an upper bound on the number of cards for which the trick can be performed.
\end{problem}

Consider a collection of $n$~unique objects.  It is trivial to bijectively map these objects to the integers $\{1,\ldots,n\}$. If five integers from the set are selected of which four are revealed, then the previous problem's trick would allow us to communicate the fifth object as long as $n$~is less than or equal to ${4! + 4 = 28}$.  Given a standard deck contains $n=52$ cards, the number of permutations is insufficient to uniquely encode the hidden card.  Therefore, we need a more creative encoding....

The key to this problem is that Esmerelda \textit{chose} which of the five cards will remain hidden.  If she chose that card randomly, we are back to 24~permutations.  However, Esmerelda was more cunning and chose a \textit{specific} card to remain hidden.  A scheme could be developed that uses this selection process to increase the upper bound on the number of encodable objects fivefold i.e.,~once for each card the volunteer chose.  When we combine that with the trick from the previous problem, we see that the maximum number of objects that can be uniquely encoded is now~${5\cdot4!=120}$.  Therefore, combining the maximum size of the encoded objects pool with the four cards revealed, we see the upper bound for the maximum deck size is~${4 + 120 = \boxed{124}}$.

The problem does not require that we prove the upper bound is the true bound.  However, after completing the problem, I did research online and found that this is in fact the true bound.


\end{document}

