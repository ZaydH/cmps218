\begin{problem}{2}{9}{33}
  Two points are selected at random on a straight line segment of length~1.  What is the probability that a triangle can be constructed out of the three resulting segments?
\end{problem}

For two randomly selected points,~$0 \leq x_1 \leq 1$ and $0 \leq x_2 \leq 1$, define $L_{\max}$ as:

\[L_{\max} = \max\bigg\{\min\{x_{1},x_{2}\},~\abs{x_{2} - x_{1}},~1-\max\{x_{1},x_{2}\} \bigg\}\text{.}\]

By the triangle inequality theorem, it is clear that a triangle can only be formed from these three resulting segments if ${L_{\max} < \frac{1}{2}}$.  Therefore if $x_{1} < \frac{1}{2}$, then $\frac{1}{2} < x_2 < x_{1} + \frac{1}{2}$.  Similarly, if $x_{1} > \frac{1}{2}$, then $x_{1} - \frac{1}{2} <  x_2 < \frac{1}{2}$.

Consider the tuple ${(x_1,x_2)}$. Each point in the domain $[0,1]^2$ is equally likely since $x_1$ and $x_2$ are selected (uniformly) at random.  Figure~\ref{fig:problem2.9.33} shows the portions of this domain (highlighted in light blue) where the lines segments induced by $x_1$ and $x_2$ can be used to form a triangle.  Note that these shaded regions correspond to the valid ranges for $x_1$ and $x_2$ derived from the triangle inequality above.

\begin{figure}[h]
  \centering
    \scalebox{0.6}{ % Scaling factor for the plot.
  \begin{tikzpicture}
  \begin{axis}[
  axis line style = thick,
  axis lines=middle,
  xmin=0,
  xmax=1,
  ymin=0,
  ymax=1,
  xlabel=$x_1$,
  ylabel={$x_2$},
  ]
  \addplot[name path=half, smooth, color=blue, domain=0:1] {0.5};
  \addplot[name path=f_1, smooth, color=blue, domain=0:0.5] {0.5 + x};
  \addplot[name path=f_2, smooth, color=blue, domain=0.5:1] {-.5+x};
  \draw [color=blue, thick] (0.5,0) -- (0.5,1);
  \path[name path=axis] (axis cs:0,0) -- (axis cs:1,0); % X axis for fill between
  \addplot [
  thick,
  color=blue,
  fill=blue,
  fill opacity=0.05
  ]
  fill between[
  of=f_1 and half,
  soft clip={domain=0:.5},
  ];
  \addplot [
  thick,
  color=blue,
  fill=blue,
  fill opacity=0.05
  ]
  fill between[
  of=f_2 and half,
  soft clip={domain=0.5:1},
  ];
  \end{axis}
  \end{tikzpicture}
}
  \caption{Regions of the domain where the line segments induced by $(x_1,x_2)$ form a triangle.}\label{fig:problem2.9.33}
\end{figure}

Since the probability of selecting any point from the above domain is uniform, the probability of selecting points that form a triangle is found via:

\begin{aligncustom}
  \Pr(\text{Form a triangle}) &= \frac{\text{Shaded Area}}{\text{Total Area}}\\
                              &= \frac{2 \cdot \left(\frac{1}{2}\right)^{3}}{1}\\
                              &= \boxed{\frac{1}{4}}\text{.}
\end{aligncustom}